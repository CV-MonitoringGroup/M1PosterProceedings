\section{実験(案1)}
(各画像に1種類の物体が一つずつでてくる(例:ホッチキスだけの画像))
\par
ハサミ, スティック糊, カッター, ホッチキス, 容器の5種類の物体を含んだ画像200枚を対象にトピック分類の性能評価実験を行う.
\par
K平均法でのクラスタ数$T$は100とした. また, LDAでのハイパーパラメータ$\alpha$は1.0, $\beta$は0.1とした.
実験の結果~ということが言える.


\section{実験(案2)}
(各画像に1種類以上の物体が一つずつでてくる(例:ハサミとカッターの画像, ホッチキスだけの画像)
\par
ハサミ, スティック糊, カッター, ホッチキス, 容器の5種類の物体を含んだ画像200枚を対象にトピック分類の性能評価実験を行う.
\par
実験の結果~ということが言える.


